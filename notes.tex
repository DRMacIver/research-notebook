\documentclass[a4paper]{book}

%% Language and font encodings
\usepackage[english]{babel}
\usepackage[utf8x]{inputenc}
\usepackage[T1]{fontenc}

%% Sets page size and margins
\usepackage[a4paper,top=3cm,bottom=2cm,left=3cm,right=3cm,marginparwidth=1.75cm]{geometry}

%% Useful packages
\usepackage{amsmath}
\usepackage{amssymb}
\usepackage{amsthm}
\usepackage{graphicx}
\usepackage[colorinlistoftodos]{todonotes}
\usepackage[colorlinks=true, allcolors=blue]{hyperref}
\usepackage{listings}


\usepackage{mathtools}
\DeclarePairedDelimiter{\ceil}{\lceil}{\rceil}
\DeclarePairedDelimiter{\floor}{\lfloor}{\rfloor}

\newtheorem{proposition}{Proposition}
\newtheorem{lemma}{Lemma}
\newtheorem{corollary}{Corollary}
\newtheorem{theorem}{Theorem}

% lstlisting macros
\lstset{
  language=Python, % choose the language of the code
  showspaces=false, % show spaces adding particular underscores
  showstringspaces=false, % underline spaces within strings
  showtabs=false, % show tabs within strings adding particular underscores
  tabsize=8, % sets default tabsize to 2 spaces
  captionpos=b, % sets the caption-position to bottom
  mathescape=true, % activates special behaviour of the dollar sign
  basicstyle=\footnotesize\tt,
  columns=fullfelxible,
  xleftmargin=2em,
  %commentstyle=\rmfamily\itshape,
  morekeywords={},
  escapeinside={(*@}{@*)},
  numbers=left
}

\title{Research Notebook}
\author{David R. MacIver}

\begin{document}

\maketitle

This is mostly a collection of research level material I am interested in.
It's something like a lab book,
something like a collection of papers I find interesting and want to make sure I remember and/or understand.

It is likely to be terribly organised.
It's primarily intended as a place for me to write stuff that is useful for me to have written down,
and only secondarily intended for public consumption.

It's also likely to be terribly formatted.
I'm not very good at \LaTeX.

\tableofcontents

\chapter{Literature Review}

This chapter is ``mostly'' literature review,
with the occasional sprinkling of my own work and ideas in it as they seem to fit.

\section{Inference of Finite Automata}\label{sec:langinference}

The paper I learned the most about this subject from was Rivest and Schapire's ``Inference of Finite Automata Using Homing Sequences''~\cite{DBLP:journals/iandc/RivestS93}.
This builds on Dana Angluin's classic L* search from ``Learning Regular Sets from Queries and Counterexamples''~\cite{DBLP:journals/iandc/Angluin87},
but extends it in several ways.

One very important feature of this paper which seems to be widely missed among people using L* search is that it significantly improves the algorithmic complexity of the number of queries that need to be performed.

Given some unknown language \(\mathcal{L}\) over an alphabet \(\mathcal{A}\),
L* search tries to find a deterministic finite automaton (DFA) representing that language given two oracles:
The first is a query oracle,
which allows you to test whether any string is in the language or not,
and the second is a counter-example oracle which allows you to test whether a given DFA is correct,
and if not presents a string for which it gives an incorrect answer.

It does this by constructing a deterministic finite automaton whose states are uniquely labelled by strings in some prefix-closed set \(S\),
and whose transitions are determined by some set of experiments \(E\).
Given a state \(s\) and a character \(c\),
the constructed automaton has a transition \(s \to t\) with \(c\) if \(sce \in \mathcal{L} \iff te \in \mathcal{L}\).
This works because of the following observations:

\begin{itemize}
\item Given \emph{any} deterministic finite automaton,
states can be uniquely labeled by a string that reaches that state from the origin.
By constructing these labels iteratively we can ensure that the set of labels used is prefix closed.
\item If two strings lead to the same state,
then any extension of them by the same string must also lead to the same state.
\item The same state cannot both be accepting and non-accepting.
\end{itemize}

Thus the experiments serve as a way of ``distinguishing'' two states:
If we ever observe that \(se \in \mathcal{L} \neq s'e \in \mathcal{L}\) then these can't lead to equivalent states.

This is the idea of the Myhill-Nerode theorem~\cite{nerode1958linear}\footnote{Confession:
I have not read the cited paper,
only the wikipedia page about it.
}.

L* search is essentially a way of learning the Myhill-Nerode automaton that results.
It proceeds in two parts:
The first tries to expand \(S\),
the second to expand \(E\).

Given a fixed set of experiments we first \emph{complete} the automaton:
For each \(s \in S\) and each \(c \in \mathcal{A}\) it tries to find a state in \(S\) that the current set of experiments thinks is equivalent to \(sc\).
If it finds one (and it cannot find more than one,
because we make sure to only enlarge \(S\) by adding inequivalent states to it),
we add a transition to that state.
If not,
we add \(sc\) to \(S\).

Once this is done,
we have an automaton that we hope matches \(\mathcal{L}\),
and we ask our counter-example oracle whether we're right.
If not,
we get a string which it gives the correct answer for.

This is where Angluin's original approach differs from Rivest and Schapire's.
Their observation builds on the surprisingly powerful idea that binary search has nothing to do with ordered sequences,
but is instead just about finding some point at which a function changes.\ 
i.e.\ if \(f(0) \neq f(n)\),
binary search finds some \(0 \leq i < n\) such that \(f(i) \neq f(i + 1)\).
If \(t\) is our counter-example then we can then use this to construct a new experiment to add to \(\mathcal{E}\) in \(O(\log(n))\) steps,
where \(n = |t|\).
We do this by taking \(f(i) = s_i t_{i+1} \ldots t_{n} \in \mathcal{L}\),
where \(t = t_1 \ldots t_n\) and \(s_i\) is the element of \(\mathcal{S}\) that labels the state that we are in after transitioning from the origin through
\(t_1, \ldots, t_i\).
We know that \(f(0) \neq f(n)\),
because \(f(0) = t \in \mathcal{L}\) and \(f(n) = s_n \in \mathcal{L}\) is the value that our automaton predicted for \(t \in \mathcal{L}\),
which was wrong.
This means that if \(i\) is our change point as above,
the transition we took from \(s_i \to s_{i + 1}\) was incorrect,
and \(t_{i + 2} \ldots t_n\) witnesses this fact,
so this is what we add to \(\mathcal{E}\).

In contrast,
Angluin's original algorithm tries to maintain \(\mathcal{E}\) as suffix-closed,
and as a result produces a much larger experiment set in a larger number of queries.

There is a lot of other information in this paper I need to go back and process at some point.
I should also write up its diversity based representation,
as I think it's a nice framing of this.

\section{DFA Minimization}

``Fast brief practical DFA minimization''~\cite{DBLP:journals/ipl/Valmari12} is a really nice paper about this,
although the experience of reading code golfed C++ to fit in a two column format wasn't a huge amount of fun.

It makes use of partition refinement,
applied to not just the states but the transitions,
beginning with a very coarse paritition that treats everything as equivalent and then progressively refining the partition whenever it finds an inconsistency.

One thing I thought was interesting about it is the core algorithm \emph{looks} like one you should have to iterate to a fixed point,
but actually it completes in a single run.

\section{Coupon Collecting and Sundry}\label{sec:coupons}

I've become interested in the non-uniform coupon collector problem recently,
which takes the following form:

Given \(X\) taking values in \(\{1, \ldots, n\}\) with \(P(X = i) = p_i > 0\),
if we have infinitely many independent copies \(X_i\) of \(X\),
and \(T = \min\limits_k |\{X_1, \ldots, X_k\}| = n\) (i.e. \(T\) is the first point we have seen every value at least once)
what is \(E(T)\)?

I proved some interesting lower bounds on this in terms of expectation that I haven't seen elsewhere but are probably not novel (I may add them in here later),
but more importantly ``Birthday Paradox, Coupon Collectors, Caching Algorithms and Self-Organizing Search''~\cite{DBLP:journals/dam/FlajoletGT92} is a really nice paper about this.

It takes the observation that this and many similar problems can be framed in terms of regular languages,
and that by using standard constructions of generating functions for regular languages,
you can more or less automate the calculation of their expected value (for values of ``automate'' that include calculating some potentially nasty integrals).

\section{Sandsifter}

Sandsifter~\cite{sandsifter} is a really neat trick that tries to find interesting hidden behaviour on x86 by just executing all the instructions and seeing what happens
(for values of ``seeing what happens'' that includes e.g.\ differential testing).
This seems infeasible:
Instructions may be up to 15 bytes long and there's no way to just try all \(256^{15} \approx 10^{36}\) such byte strings.

But it turns out that it's not!
The reason is that although they may be that long,
the instruction set is \emph{prefix-free}---no
instruction is a prefix of another.

This means that the following algorithm enumerates the entire instruction set:

\begin{enumerate}
\item Start from an all-zero buffer of 15 bytes.
\item Repeatedly execute the current buffer and see how many bytes are consumed.
\item Zero everything after the last byte read and increment the last byte read (handling overflow by setting it to zero and incrementing the previous one)
\item Stop when we have executed an instruction where every executed byte was 255.
\end{enumerate}

Because even though \emph{some} instructions are 15 bytes long \emph{most} aren't,
and in fact there turn out to be only about \(100,000,000\) instructions.

The only problem that remains is how to determine how many bytes were read given an x86 instruction.
This isn't something that the CPU tells you.
The answer is really neat:
Put the instruction right before a page boundary and watch to see if you trigger a fault when executing it!
If you do, then the instruction tried to read off the end of the page.
By sliding the position of the instruction around you can then find the smallest prefix that does not trigger a page fault.

\section{Parsing with Derivatives}

Parsing with derivatives\cite{DBLP:conf/icfp/MightDS11, DBLP:conf/pldi/0001HM16} is a really neat concept about how you can use laziness and meomization to extend the Brzozowski Derivative to context free languages,
essentially building up an infinite state automaton lazily as you traverse it.

I'm quite interested in applying this to the generation of objects in context free grammars,
but so far mostly just haven't---some
initial prototypes were reasonably promising but have never really manifested into anything very interesting.

One place this \emph{does} get used in Hypothesis is that the improved fixed point calculations in~\cite{DBLP:conf/pldi/0001HM16} are how Hypothesis calculates various properties of strategies which may recursively refer to each other.

\section{Boltzmann Sampling}

Given some combinatorial class of objects \(\mathcal{C}\) with a notion of size,
we might want to sample from it in such a way that any two objects with the same size occur with the same probability.
Boltzmann Samplers are a class of samplers which have this property and are amenable to mostly automated constructions.

The idea is that we consider the generationg function \(f(z) = \sum\limits_n z^n C_n\),
where \(C_n =  |\{c \in C: |c| = n\}|\}\),
so the coefficient of \(z^n\) is the number of elements of \(\mathcal{C}\) with size \(n\).

The Boltzmann samplers for \(\mathcal{C}\) are then a family of samplers with parameter \(z\) which pick an element of size \(n\) with probability \(z^n \frac{C_n}{f(z)}\).

The nice thing about this is that they combine well.\ 
e.g.\ if we have two disjoint classes of \(\mathcal{C}\) and \(\mathcal{C}'\) then we can construct a Boltzmann sampler for \(\mathcal{C} \cup \mathcal{C}'\) as a weighted choice between the Boltzman samplers for each,
picking from \(\mathcal{C}'\) with probability \(\frac{f'(z)}{f(z) + f'(z)}\).

A neat thing I have observed\cite{falbs} that probably isn't novel but that I haven't seen elsewhere is that given a regular language expressed as a deterministic finite automaton
(e.g.\ by using derivatives!)
you can automatically calculate a Boltzmann sampler for it using symbolic linear algebra
(or non-symbolic linear algebra if you fix the parameter up front).

The similarity of this paper to~\cite{DBLP:journals/dam/FlajoletGT92} is probably not accidental given that they have an author in common---both
are about using generating functions of regular languages to simplify some natural probabilistic problem through semi-automatic constructions.

\section{Fault Independence}

A nice set of papers starting from ``An Experimental Evaluation of the Assumption of Independence in Multiversion Programming''~\cite{DBLP:journals/tse/KnightL86}.

Key ideas:

\begin{itemize}
\item We want to see if multiple independent implementations of the same spec tend to fail in the same way.
\item Get a number of different people to implement the spec.
\item Then feed their implementations a million random inputs designed to look like a ``realistic workload'' and see if the inputs on which they fail are statistically independent.
\end{itemize}

Spoiler alert: They're not.

I haven't read the paper in detail yet,
but I \emph{have} read ``A reply to the criticisms of the Knight \& Leveson experiment''\cite{knight1990reply} which as well as being an intersting summary of the results is an absolutely glorious takedown.

As well as really liking the empiricism of this work,
one reason I'm interested in it is that a lot of the testing literature's focus on differential testing should be considered to have a fairly big hole down the middle as a result of this work---if everyone is making the same mistakes,
then differential testing won't pick this up,
and if mistakes are non-independent then the probability of this happening may be much larger than expected.

\section{Coverage Guided Fuzzing}

I need to do more research on this before I have anything intelligent to say about it,
but here are some things I think are neat ideas.

American Fuzzy Lop (AFL)~\cite{AFL} is of course kinda a big deal in this space.
There's quite a lot of literature on extending it,
though I'm not wholly convinced by it.

Nezha~\cite{DBLP:conf/sp/PetsiosTSKJ17} has the core idea that if you're doing differential testing between a bunch of different versions of something then maybe you should prioritise things that exhibit differences in behaviour!
It starts from some measurement \(\delta(x, s)\) taking an input and an SUT and returning some label for it,
then when doing differential testing between \(s_1, \ldots, s_n\) you can associate with an input \(x\) the profile \(\delta(x, s_1), \ldots, \delta(x, s_n)\).
Any particular value of this profile is not ``intrinsically'' interesting,
but the fact that two different inputs give a different profile indicates that the SUTs are exhibiting some difference in behaviour.
The important part of this is that they can do this without showing up any particularly novel behaviour in any one SUT!\@
I have not looked over their claims of its efficacy in any detail,
but it's an interesting idea.

Cause Reduction~\cite{DBLP:journals/stvr/GroceAZCR16} is the idea,
dear to my heart,
that you can apply test case reduction to many other problems,
not just bugs.
In particular you can apply it to coverage targets!
In this case they applied it mostly to coarse grained coverage targets (lines, branches, etc),
but I'm also interested in what happens with more fine grained ones.

\chapter{Miscellaneous Sums and Integrals}

The following came up in~\cite{DBLP:journals/dam/FlajoletGT92} as ``well known''.
It wasn't well known to me,
so I felt obliged to calculate it.
I'm not sure the process was enlightening enough to be worth it.

\begin{proposition}
\(\sum\limits_{q = 1}^m {(-1)}^{q - 1} {m \choose q} \frac{1}{q} = H(m)\)
\end{proposition}

\begin{proof}
\begin{align*}
\sum\limits_{q = 1}^m {(-1)}^{q - 1} {m \choose q} \frac{1}{q} &= \sum\limits_{q = 1}^m {(-1)}^{q - 1} {m \choose q} \int\limits_0^1 x^{q - 1} dx\\
&= \int\limits_0^1 \sum\limits_{q = 1}^m {(-1)}^{q - 1} {m \choose q} x^{q - 1} dx\\
&= \int\limits_0^1 -x^{-1} \sum\limits_{q = 1}^m {m \choose q} {(-x)}^q dx\\
&= \int\limits_0^1 -x^{-1} \left( \sum\limits_{q = 0}^m {m \choose q} {(-x)}^q - 1 \right)dx \\
&= \int\limits_0^1 -x^{-1} \left( {(1 - x)}^m - 1 \right)dx \\
&= \int\limits_0^1 {(1 - x)}^{-1} (x^m - 1) dx \\
&= \int\limits_0^1 \sum\limits_{n = 0}^\infty x^n (x^m - 1) dx \\
&= \sum\limits_{n = 0}^\infty \int\limits_0^1 x^n (x^m - 1) \\
&= \sum\limits_{n = 0}^\infty \frac{1}{n + m} - \frac{1}{n} \\
&= \lim\limits_{k \to \infty}  H(m) - \sum\limits_{n = k}^{m + k} \frac{1}{n + m}\\
&= H(m)\\
\end{align*}

Where the conversion of the infinite sum to the limit works because for every \(n \geq m\) the positive term in the summand cancels the negative term from the summand for \(n - m\),
and the final limit is calculated because \(0 \leq \sum\limits_{n = k}^{m + k} \frac{1}{n + m} \leq \frac{m}{n}\).
\end{proof}

\chapter{Miscellaneous Research Questions}

\section{Lower Bounds on Collecting Coupons}

Given the setup of Section~\ref{sec:coupons} for the non-uniform coupon collector problem,
here is an investigation into lower bounds on the expected waiting time for a complete set of coupons.
There are many like it, but this one is mine.

\begin{lemma}\label{lemma:couponpartition}
Let \(U_i\) be the set of distinct values seen by time \(i\) and for \(A \subseteq \{1, \ldots, n\}\) let \(S^A\) be the number of times we have seen \(X_i \in A\) before we have seen every value in \(A\).
Note that in terms of probabilities this is the same as being \(S\) for the conditional distribution \(X | X \in A\),
but it's important that these are actually using the same \(X_i\).

Then \(E(S|U_i = A) = i + \frac{1}{1 - P(X \in A)} E(S^{A^c})\)
\end{lemma}

\begin{proof}
Let \(L_0 = i\) and \(L_{k + 1}\) be the first time after \(L_k\) that \(X_{L_{k + 1}} \not\in A\).

Then \(E(S | S^{A^c} = m) = E(L_m)\).
But \(L_{k + 1} - L_{k}\) is geometrically distributed with parameter \(P(X \not\in A) = 1 - P(X \in A)\),
so \(E(L_{k + 1} - E_(L_k)) = \frac{1}{1 - P(x \in A)}\),
and so \(E(L_m) = i + \frac{m}{1 - P(x \in A)}\).
By applying conditional expectation we thus have \(E(S) = i + \frac{1}{1 - P(X \in A)} E(S^{A^c})\) as desired.
\end{proof}

This partition result also gives u

\begin{theorem}
\(E(S) \geq n H(n)\), with equality if and only if \(X\) is a uniform distribution.
\end{theorem}

\begin{proof}
We will do this by induction.
It is trivially true when \(n = 1\) because \(S = 1\) in that case.

By applying the previous lemma,
\(E(S | X_1 = i) = 1 + \frac{1}{1 - p_i} E(S^{{\{i\}}^c})\).

\(S^{{\{i\}}^c}\) has the same distribution as \(S\) for the conditional distribution of \(X | X \neq i\),
so in particular it has the same expecation.
Thus by our inductive hypothesis,
\(E(S^{{\{i\}}^c}) \geq n H(n)\) with equality if and only if this conditional distribution is uniform.

Let \(b_i = E(S^{{\{i\}}^c})\).
Then by conditional expectation,
\(E(S) = 1 + \sigma \frac{p_i}{1 - p_i} b_i \geq (n - 1) H(n) \sigma \frac{p_i}{1 - p_i} \).

This sum can be seen to be strictly minimized when \(p_i = \frac{1}{n}\),
which gives us \(E(S) \geq 1 + (n - 1) H(n) \frac{n}{n - 1}\) with equality if and only if \(X\) is uniform.

A simple induction shows that this term is equal to \(n H(n)\) and the result is proved.
\end{proof}

\begin{proposition}\label{prop:coupontail}
If \(p = P(X \geq m)\) then \(E(T) \geq \frac{n - m + 1}{p}\).
\end{proposition}

\begin{proof}
This is just the lemma applied to the very crude lower bound that \(E(S^A) \geq |A|\).
\end{proof}

\begin{proposition}\label{prop:couponestimate}
For any strictly increasing function \(f\) and \(1 \leq m \leq n\)
\(E(S) \geq \frac{f(m) (n - m + 1)}{E(f(X))}\).
\end{proposition}

\begin{proof}
This follows straightforwardly from the standard result that \(P(X \geq m) \leq \frac{E(f(X))}{f(m)}\) and Corollary~\ref{prop:coupontail}.
\end{proof}

\begin{theorem}
\(E(S) \geq \frac{2 {(k + 1)}^{-1} {(n + 1)}^{k + 1} - e (k + 1) {(n + 1)}^{k - 1}}{E(X^k)}\)
\end{theorem}

\begin{proof}
Let \(f(x) = x^k\) in Proposition~\ref{prop:couponestimate}.

Now, define \(g(x) = (n - x + 1) x^k\).
We'd like to find \(m \in \{1, \ldots, m\}\) that maximises \(g(m)\) to get the largest lower bound on \(E(S)\).

\begin{align*}
g'(x) &= (n - x + 1) k x^{k - 1} - x^k\\
&= (n + 1) k x^{k - 1} - (k + 1) x^k\\
&= \left( (n + 1) k - (k + 1) x \right) x^{k - 1}\\
&= \left( (n + 1) \frac{k}{k + 1} - x \right)(k + 1) x^{k - 1}\\
\end{align*}

So \(g'(x) = 0\) when \(x = x_{\max} = \frac{k}{k + 1} (n + 1)\).

Unfortunately we can't just plug in \(x_{\max}\),
because it might not be integral.
But we can find an integer \(1 \leq i_{\max} \leq n\) with \(|x_{\max} - i| \leq 1\).
\(|g'(y)| \leq (k + 1){(1 + x_{\max})}^{k - 1}\) for \(y\) between \(x_{\max}\) and \(i_{\max}\),
so

\(g(i_{\max}) \geq (n - x_{\max} + 1)x_{\max}^{k} - (k + 1){(1 + x_{\max})}^{k - 1}\)

\begin{align*}
(n - x_{\max} + 1)x_{\max}^{k} &= \frac{n + 1}{k + 1} {(n + 1)}^k {(1 + k^{-1})}^k\\
&= {(k + 1)}^{-1} {(n + 1)}^{k + 1} {(1 + k^{-1})}^k\\
&\geq 2 {(k + 1)}^{-1} {(n + 1)}^{k + 1} \\
\end{align*}

Where the latter bound comes from the fact that \(x \to {(1 + x^{-1})}^x\) is monotonic increasing.

\begin{align*}
{(1 + x_{\max})}^{k - 1} &= {(n + 1)}^{k - 1} {(1 + k^{-1})}^k \\
&\leq e {(n + 1)}^{k - 1}\\ 
\end{align*}

Thus we have \(g(i_{\max}) \geq 2 {(k + 1)}^{-1} {(n + 1)}^{k + 1} - (k + 1) e {(n + 1)}^{k - 1}\).
Plugging this in to the results of the previous corollary,
this gives us the desired result.

\end{proof}

\begin{corollary}
If \(X\) comes from some family of distributions such that \(E(X^k)\) is bounded as \(n\) varies,
then \(E(S) \geq O(n^{k + 1})\).
\end{corollary}

This bound is not quite tight,
and is missing a logarithmic factor.
A more involved calculation could ``easily'' recover it---we
simply dropped the factor of \(H(n - m + 1)\) in our initial approximation of how long it would take to fill all \(k \geq m\).

These bounds are particularly interesting because \(E(S)\) is independent under permutations of \(\{1, \ldots, n\}\),
and thus we may reorder the \(p_i\) values however we want before calculating these bounds.

\begin{theorem}
Let \(\sigma\) be a permutation of \(\{1, \ldots, n\}\).
If \(f\) is strictly increasing on \([1, n]\) then \(E(f(\sigma(X)))\) is strictly minimized when \(\sigma(i) < \sigma(j)\) implies \(p_i \geq p_j\).
\end{theorem}

\begin{proof}
Suppose that \(\sigma\) does not have this property.
Pick \(i, j\) with \(p_i < p_j\) and \(\sigma(i) < \sigma(j)\).
Now create \(\sigma' = (\sigma(i) \sigma(j)) \cdot \sigma\),
the permutation that first performs \(\sigma\) and then swaps \(\sigma(i)\) with \(\sigma(j)\).

Now:

\begin{align*}
E(f(\sigma'(X))) - E(f(\sigma(X))) &= p_i \sigma(j) + p_j \sigma(i) -  p_j \sigma(j) - p_i \sigma(i) \\
&= (p_j - p_i) (\sigma(i) - \sigma(j)) \\
&< 0\\ 
\end{align*}

Where the latter inequality follows because \(p_j - p_i > 0\) and \(\sigma(i) - \sigma(j) < 0\),
by our assumptions.
\end{proof}

\begin{theorem}
Let \(X\) be such that \(p_{i + 1} \leq p_i\),
and let \(f\) be a strictly increasing function on \(\{1, \ldots, n\}\)
Then \(E(f(X))\) is strictly maximized when \(X\) is the uniform distribution.
\end{theorem}

\begin{proof}
A similar argument as the previous theorem: Suppose we have \(i\) with \(p_{i + 1} < p_i\).
Write \(p_i = a + b, p_{i + 1} = a - b\),
where \(a = \frac{p_i + p_{i + 1}}{2}\) and \(b = \frac{p_i - p_{i + 1}}{2}\).
Then if \(X'\) results from replacing both \(p_i\) and \(p_{i + 1}\) with \(a\),
this will result in

\begin{align*}
E(X') - E(X) &= a f(i) + a f(i + 1) - (a + b) f(i) - (a - b) f(i + 1) \\
&= b (f(i + 1) - f(i)) \\
&> 0\\
\end{align*}

So whenever we have two adjacent probabilities that are not equal,
we can strictly increase the expectation by making them equal,
thus this expectation is maximized by the uniform distribution.

\end{proof}

Thus with any strictly increasing function \(f\),
once we have reorderd the values of \(X\) in this way we can regard \(E(f(X))\) as a measure of ``how far from uniform'' \(X\) is.

\section{Boltzmann Sampling of Levenshtein Automata}

\newcommand{\levlang}[2]{\mathcal{L} (#1, #2)}

Let \(\mathcal{A}\) be an alphabet.
To avoid triviality assume \(|\mathcal{A}| \geq 2\).

The Levenshtein distance of two strings \(u, v\) over \(\mathcal{A}\), \(d(u, v)\), is the length of the shortest path from \(u\) to \(v\) where a step in the path is either replacing a single character, deleting as ingle character, or inserting a single character.

The set of strings \(\levlang{u, n} = \{ v: d(u, v) \leq n \}\) is finite,
and so certainly regular,
but it even admits a fairly small DFA (this is not a new observation at all).

This means we should be able to calculate a Boltzmann sampler for it\cite{falbs}!

Can we do so efficiently?

We'll use the Brzozowski derivative to explore the structure of these languages.

Let \(b_L\) be the generating function of a language \(L\).

First: A crosscheck we can use is that the coefficients in the generating function must always be polynomials,
and the coefficients must be in \([|u| - n, |u| + n]\),
as any single step can only change the size by at most one,
so there are no strings of length outside that range to contribute to the generating function.

First,
two useful special cases:

\begin{proposition}
Now note that \(\levlang{\epsilon}{n}\) is the set of all strings of length at most \(n\),
so has generating function \(b_{\levlang{\epsilon}{n}}(z) = \frac{1 - {(|A|x)}^{n + 1}}{1 - |A| z}\).
\end{proposition}

\begin{proposition}
\(\levlang{v}{0}\) matches only \(v\),
so has \(b_{\levlang{v}{0}} = z^{|v|}\) 
\end{proposition}

\begin{theorem}
Let \(v = cu\) where \(c \in \mathcal{A}\),
and let \(\delta\) be the Brzozowski derivative.

\begin{itemize}
\item \(\delta(\levlang{v}{n}, c) = \levlang{u}{n}\).
\item \(\delta(\levlang{v}{b}) = \levlang{cu}{n - 1} \cup \levlang{u}{n - 1} \cup \delta(\levlang{u}{n - 1}, b)\)
\end{itemize}

\end{theorem}

\begin{proof}
If \(b = c\) there is nothing to do that we couldn't do later,
so we can just consume it and match the rest of the string,
thus \(\delta(\levlang{cu}{n}, c) = \levlang{u}{n}\).

If \(b \neq c\) then we can either delete \(b\) from the string,
meaning that the suffix must now be in \(\levlang{cu}{n - 1}\),
replace it with \(c\),
meaning that the suffix must now be in \(\levlang{u}{n - 1}\),
or insert \(c\),
which means that the remainder of the string \emph{including the current character},
must be in \(\levlang{u}{n - 1}\),
i.e.\ the suffix must be in \(\delta(\levlang{u}{n - 1}, b)\).
So \(\delta(v, b)\) is the union of these three possibilities.
\end{proof}

\begin{proposition}
Decompose \(v\) as \(v_1 \ldots v_m\) where each \(v_i\) consists of repetitions of a single character \(c_i\),
and the character in \(v_i\) is distinct from that in \(v_{i + 1}\).
Let \(n = |v|\) and suppose \(n > 0\).

Then \(b_{\levlang{v}{1}} = m z^{n - 1} + (1 + (|\mathcal{A}| - 1)n) z^{n} + ((n + 1) |\mathcal{A}| - n) z^{n + 1}\)
\end{proposition}

\begin{proof}
This is a relatively straightforward counting argument.

You can obtain a string of length \(n - 1\) only by deleting a single character,
but deleting any character inside \(v_i\) has the same effect,
so there are only \(m\) distinct deletions.

You can obtain a string of length \(n\) only by leaving the string unchanged or replacing a single character,
and each index admits \(|\mathcal{A}| - 1\) such replacements.

The calculation for \(n + 1\) is slightly harder.
Such a string must be obtained by an insertion,
but we must be careful not to overcount.

First,
we count all the insertions that just increase the length of one of the \(v_i\) by \(1\).
This gives us exactly \(m\) new strings.

Now, there are \(n + 1\) points we can insert at.
If these are interior to a \(v_i\)---that is,
it is between two characters of the same value---
then there are \(\mathcal{A} - 1\) possible values to insert (anything other than \(c_i\)),
but if it is the boundary then there are  \(\mathcal{A} - 2\) values---anything
other than \(c_i\) or \(c_{i + 1}\).

There are exactly \(m - 1\) points that are not interior---in
between each of the \(v_i\).

Thus there are \((m - 1)(|\mathcal{A}| - 2) + (n - m + 2)(|\mathcal{A} - 1|) = (n + 1) |\mathcal{A}| - m - n\).
Counting the insertions that expanded a \(|v_i|\),
this gives us a total of \((n + 1) |\mathcal{A}| - n\) as desired.
\end{proof}

In general I doubt the existence of a nice formula for this Boltzmann Sampler,
but the nice thing is that we can use the derivative method to simulate it lazily.
We can also use the above formula to simplify our calculations whenever we get down to a \(\levlang{v}{1}\) term.

\chapter{Patches to L*}

Here are some ideas I have for how to improve L* search.
The subtext here is that I have a bunch of problems where I would like to use it but am currently unable to due to its cost.

\section{L* Search with Character Partitioning}\label{sec:lstarcharpartition}

As observed in ``Regular-expression derivatives re-examined''\cite{DBLP:journals/jfp/OwensRT09},
it can be extremely helpful to consider DFA transitions not as happening on a single character,
but instead a whole class of locally equivalent characters.

I think L* search (see Section~\ref{sec:langinference}) could benefit from the same observation!

Suppose we modified L* search as follows:
Whenever we create a state,
we associate a partition refinement data structure with it,
initially treating all characters in the alphabet as equivalent.

When completing the transition graph,
we then only consider one character (say the smallest, or a randomly chosen one) from each partition.

When giving a counter-example,
this gives us a transition \(s \to s'\) via a character \(c\),
and a new experiment \(e\) which distinguishes \(s'\) from the state that we \emph{should} have transitioned to.
Note that unlike in conventional L*,
that state might be in the table already!

We can now use this experiment to refine the partition that \(c\) belongs to,
by evaluating \(sc'e\) for every \(c'\) that we currently believe is equivalent to \(c\).
If they do not all give the same answer,
this splits the partition.
Note that in this case we do not need to add the experiment to our set of experiments!

This is still a correct algorithm for inference because it reduces to L* in the case where all partitions are fully refined,
and at each point where we invoke the teacher to give us a counterexample one of three things happens:

\begin{enumerate}
\item The teacher tells us our DFA is correct and we are done.
\item We find a new inequivalent state.
\item We strictly refine one of the partitions.
\end{enumerate}

The reason is that either the canonical element we chose for the partition also produces a wrong outcome,
in which case a new state is found as it usually is in L* search,
or \(c\) exhibited different behaviour with \(e\) than said canonical element,
so the partition is strictly refined.

Thus whenever the DFA is not correct,
we make progress,
and thus we eventually terminate in a correct answer.

How much of a complexity improvement is this?

Well, it depends.
It's not a strict improvement,
because we make more calls to the teacher,
but we potentially save a huge amount of time on query calls.

Suppose the minimal DFA has \(m\) states and the alphabet has size \(n\).
Let \(d_i\) be the number of equivalence classes of characters starting from state \(i\).
Equivalently,
\(d_i\) is the number of states reachable in a single transition from \(i\).

Then when we have \(k\) experiments,
classically L* search requires \(O(mnk)\) queries to complete its transition table (once we have found some suitable fraction of the available states).
This changes that to \(O(k \sum d_i)\).
In particular,
\(d_i \leq m\),
so we now only need to perform \(O(km \min(m, n)) \) queries to get completion (and possibly much less).

However,
we may have to perform an additional \(O(\sigma d_i)\) queries to the teacher,
because we now have refutation steps that only increase the number of refinements.

This might be fine.\
e.g.\ for my use case more queries to the teacher is in some sense what we \emph{want}---if
the idea here is that we are using the DFA to guide some fuzzing process then we'd ideally spend almost \emph{all} the time querying the teacher,
and the table rebuilding process is something we only do when things go wrong,
and performing \(256 m\) queries each time is hugely undesirable.

\section{L* with guessing}

I haven't quite pinned down what I mean by this,
but I have this intuition that the big problem with L* is that it is too afraid of looking bad in front of teacher.
That is,
it tries too hard to get the right answer,
incurring a large cost in number of queries to avoid a small cost in number of requests for counter-examples.
If you're in a situation like I am where counter-examples are either cheap or uninteresting,
this isn't a good trade-off.

So what if you just guessed an answer that was likely to be correct when doing so is cheap?

For example, imagine we have some plausible ordering \(e_1, \ldots, e_k\) of the experiments,
and we want to calculate which state \(s\) corresponds to.
Say discovery order + self-organising search or something.

Now suppose we want to determine the state corresponding to \(t\).
We run down our experiments,
repeatedly pruning the list of possible states.

What we are \emph{supposed} to do is to run every experiment,
but what if we just stopped when we got down to one state?

The answer is of course we would make more wrong transitions,
but maybe that's fine?

When our teacher points out that a transition was wrong,
we can just check if it was based on a guess,
and before we do anything else we run the rest of the experiments to see whether we already knew this should have been a novel state before adding a new experiment to the list.

\section{L* with local experiments}\label{sec:lstarlocal}

One of the big problems with L* is that it has to query every experiment against every state.
This gets expensive fast!

What if we instead maintained it as a decision tree?
Each state corresponds to a leaf in the tree,
and when we find an experiment that distinguishes some previously believed to be equivalent state from it,
we split the leaf into a branch.

The tree has the potential to become horribly unbalanced,
but the worst case scenario is that we perform exactly as many queries as we were going to before!

Although that's not technically true without a modification:
This could cause us to add more experiments than we previously would have.
When we discover a state needs to split,
perhaps we should first try the experiments that we've previously seen that were not on the path to that state before adding a new one?

\section{Recursive L*}

This is currently a \emph{very} half-baked idea (and if you're not me and reading this section it probably won't make much sense).

We have some prefix free language \(\mathcal{L}\), a classifier \(f: \mathcal{L} \to \mathbb{N}\),
and a random variable \(X \in \mathcal{L}\).
We would like to construct the conditional random variable \(X|f(X) = i\) more efficiently than using rejection sampling
(as we're specially interested in doing this in cases where \(P(f(X) = i)\) is small).

We are interested in particular in the language \(\{x: \exists i, P(X = i|X \text{ starts with } x) = 1\}\)---these
are the set of prefixes which are enough to fully define the behaviour of \(X\).
Note that unlike \(\mathcal{L}\) this is very much not prefix-free:
It is closed under taking extensions!

The idea I have been trying to figure out is whether we can do this with language inference,
by creating and then simulating a deterministic finite automaton based on the language.
This is not really viable,
both because language inference is intrinsically expensive (even with some of the above ideas) and because the structure of the language while technically regular (we have a size upper bound if we really need one) is potentially massively complicated.

The idea I'd like to explore in this section is whether we can make use of some of our knowledge about the structure and boundary information of the language to fix this.

In Hypothesis,
we keep track of example boundaries.
This has a couple of consequences:

\begin{itemize}
\item We can tell exactly which substring corresponded to a given draw call.
\item We can perform \emph{precise random replacements} of any draw call we choose---we
draw the same bytes up to that point and while the draw call of that index is active we return random bytes,
when it becomes inactive we read from the prefix.
\end{itemize}

Note that for any draw call,
the set of strings corresponding to the draw at that index in any test case which agrees with the prefix of this one is itself a prefix free language.

This means that at any index that is right after a completed draw call (counting bytes as draw calls here),
we have a prefix free language of bytes that can come next (and a generator for them!).

So we can consider this as a prefix-free regular language where the alphabet elements are not bytes,
but are themselves other prefix-free regular languages!

We have no hope of performing full blown L* here,
but maybe we don't need to!

We identify states not with strings as in regular L* but with a TestResult object plus an index into its draws and bytes calls.

A state may be in one of two err states:
Restricted and unrestricted.
Unrestricted basically means that we don't think the value emitted here matters.
We want as many states to be unrestricted as possible.
Unrestricted states are cheap.

A restricted state is one we've started building an automaton for,
recursively applying the process to the local language!

At the top level we work with the classifier that evaluates \(f(x0\ldots)\).
Either this will successfully measure whether starting with \(x\) always gives the same value of \(f\),
or we'll eventually find some string where it doesn't anyway.

When we recursively apply this to the language of a given state,
we instead apply it to the function that similarly zero-extends the string and then look at the state we end up in in the current state machine.

We can then generate a test case by recursively walking the state graph:
If the current state is unrestricted,
just generate it.
If it's restricted,
generate a random walk through its state graph that ends up with the desired classification,
then fill in by recursively generating an example from each of the transitions made.
Finally, because we only matched up to some prefix, extend the rest with random noise.

We can now check the results against our desired outcome.
If it agrees with it, great we're done.
If not we now begin a correction procedure.

In one sense this is much easier than L*,
because we know the exact point at which things wrong---where
we entered an accepting state.

But there are two possibilities here:
One is that we need to introduce a new state,
but another is just that our generator for the sub-language put us in the wrong place.
So first we have to check what state we actually ended up in.
If it's the right one,
then we add the experiment to our set and split off a new state.
Either way, we then recursively apply the correction procedure to the language for this state.

\section{Lazy L*}

I think this might be the idea that ties everything together.

The key notion here is that actually I don't care about the goal of L*---I
don't expect languages to be regular,
and I don't much care about their DFA per se.

This is false in that having a DFA of known structure would be extremely useful,
but given that I don't expect there to be one I think it makes sense to try to figure out how to do with something weaker.

The construction phase of L* can be thought of as separated out into two parts:

\begin{enumerate}
\item A process of partition refinement of the infinite set of strings\footnote{
As in Section~\ref{sec:lstarcharpartition} this is also a useful way of thinking about the behaviour at any given state.
}.
\item The construction of a DFA from partition refinement.
\end{enumerate}

For the partition refinement,
what we specifically have is that we have some infinite set \(X\) and a set of classifiers \(f_i: X \to \mathbb{N}\) and we want to get progressively better approximations to the partitions defined by \(p_x = \{y: \forall y, f_i(x) = f_i(y)\}\).

Suppose we had an partition refinement data structure that looked as follows:

\begin{lstlisting}
class PartitionRefinement(object):
  def __len__(self):
    """The number of currently known distinct partitions."""

  def __getitem__(self, i):
    """The canonical representative for partition i.
    This will never change - once a canonical representative
    is chosen it is fixed."""

  def partition_of(self, value):
    """Return the current partition of value. If it is not
    in any of the existing partitions, a new one will be
    created and value will be used as its canonical representative.
    """

  def split_partition(self, value, classifier):
    """Let i = self.partition_of(value). If classifier(value)
    != classifier(self[i]) then create a new partition with
    value as its canonical representative, and split the current
    partition so that any values in it with distinct outcomes
    for classifier(v) go in different partitions."""
\end{lstlisting}

L* search then implies a specific concrete implementation of this data structure that performs every operation fully strictly:
When you call \texttt{split\_partition} it appends it to a global list of experiments,
and when you look up a string it consults every one of those experiments to refine the partition.
However, other implementations are possible!
e.g. the local experiments idea in Section~\ref{sec:lstarlocal} is basically a suggestion of an alternate implemntation of this idea.
Additionally,
this API allows us to calculate the refinements \emph{lazily}.
We don't have to determine if \(s\) should actually go in a new partition until we're asked what partition \(s\) is in.

In L* we then build the DFA transitions eagerly as follows:

\begin{lstlisting}
def build_dfa(partition_table, characters):
  # This should have been set up at the beginning
  assert len(partition_table) > 0
  assert partition_table[0] == b''

  transitions = []

  # we might increase the number of partitions as we
  # go, so this can't be a normal for loop.
  i = 0  
  while i < len(partition_table):
    assert len(transitions) == i
    transitions.append([])
    s = partition_table[i]
    for c in characters:
      transitions[i][c] = partition_table.partition_of(s + c)
    i += 1
  return transitions
\end{lstlisting}

But there's no reason we have to build it eagerly!
One of the neat things about the implementation of regular expression matching in terms of the Brzozowski derivative is that you can build the DFA lazily as you match,
and we can do exactly the same thing here.

This is particularly useful if we want to somehow extract a subset of the automaton that is much smaller than the larger automaton!

For example,
suppose we had some set of strings \(U\) and we want to construct the automaton consisting of only states that are on the path for some \(x \in U\).
We can simply enumerate \(U\) and walk the automaton,
noting which states appear there,
and extract the subset of the automaton that contains only those states.

\chapter{Reading List}

Some things I should get around to reading but currently haven't,
in no particular order of priority.

This is a rather crude way of bookmarking papers and books.
It's mostly to get them into the bibtex and somewhere on permanent record rather than leaving them in Browser tabs waiting to be lost.


\begin{itemize}
\item More of the work of \href{Nancy Leveson}{http://sunnyday.mit.edu/papers.html}
\item The work of \href{http://mcs.open.ac.uk/hcs2/}{Helen Sharp}.
\item Dilemmas in a general theory of planning\cite{rittel1973dilemmas}.
\item Computers and language learning: An overview\cite{warschauer1998computers}
\item Nonparametric estimation of the number of classes in a population\cite{chao1984nonparametric} (probably going to be more interesting to read things that cite it, as this is quite old, but might be worth understanding the paper first).
\item Royen’s proof of the Gaussian correlation inequality\cite{latala2017royen}
\item Five stages of accepting constructive mathematics\cite{bauer2017five}.
\item Birds of the Internet: Towards a field guide to the organization and governance of participation\cite{fish2011birds}.
\item Mechanizing exploratory game design\cite{smith2012mechanizing}.
\item Can a machine design?\cite{cross2001can}
\item A bunch of work by \href{http://www.robertfeldt.net/}{Robert Feldt} and Simon Poulding on test case generation
(oh look! Something actually relevant to my PhD!).
In particular~\cite{DBLP:conf/icst/PouldingF15, DBLP:conf/issre/FeldtP13, DBLP:conf/icse/FeldtP15}.
\item Tools for Discovery, Refinement and Generalization of Functional Properties by Enumerative Testing\cite{matela2017tools}.
\item State Complexity of Boundary of Prefix-Free Regular Languages\cite{DBLP:journals/ijfcs/EomH15}
\item Probabilistic DFA Inference using Kullback-Leibler Divergence and Minimality\cite{DBLP:conf/icml/ThollardDH00}
\item Inference of regular languages using state merging algorithms with search\cite{DBLP:journals/pr/BugalhoO05}.
\item Learning Regular Languages over Large Ordered Alphabets\cite{DBLP:journals/corr/MensM15}
\item Hitting Families of Schedules for Asynchronous Programs\cite{DBLP:conf/cav/ChistikovMN16}
\end{itemize}

\bibliography{references}{}
\bibliographystyle{acm}

\end{document}
