\chapter{Updating Random Weighted Samplers}

The problem this chapter is about is trying to maintain a set of values \(x_1, \ldots, x_n\) with corresponding weights \(w_i > 0\),
such that we can efficiently add and remove values and change weights, and also efficiently sample \(x_i\) with probability proportional to \(w_i\).

The papers I learned basically everythign I know about this from are:

\begin{itemize}
\item ``Dynamic Generation of Discrete Random Variates''\cite{DBLP:journals/mst/MatiasVN03}.
\item ``Optimal algorithms for generating discrete random variables with changing distributions''\cite{hagerup1993optimal}.
This appears to be an updated (and unpublished!) version of~\cite{DBLP:conf/icalp/HagerupMM93}.
\end{itemize}

One of them is much more complicated than th e other and I no longer remember which is which.

I was told about both by \href{https://cstheory.stackexchange.com/a/37651/4624}{this CS theory stackexchange answer}.

The simpler approach I have used in the past inspired by these is to combine two data structures:

\begin{itemize}
\item Bucket the weights by \(2^n \leq w_n < 2^{n + 1}\).
\item Store the buckets in a tree which maintains total weight on its nodes and use the obvious \(\log(n)\) algorithm to pick a bucket based on the total weight in it.
\item Perform rejection sampling within that bucket (which terminates in expected at most two loop iterations because of the bucketing).
\end{itemize}
