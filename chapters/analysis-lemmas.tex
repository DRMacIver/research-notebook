\chapter{Some Lemmas (in Analysis)}

These are some old notes I had about analysis.

These are probably more indicative of what I used to care about than anything else,
but I thought it might be interesting to transcribe them (the original PDF is in the pdfs directory of this repository).

Apparently the following lemma comes up when defining topological entropy (I don't remember what topological entropy is\ldots):

\begin{lemma}
Let \(a_i\) be a sequence of real numbers such that \(a_{m + n} \leq a_m + a_n\).
Then \(\lim\limits_{a_n \to \infty} \frac{a_n}{n} = \inf \frac{a_n}{n}\).
\end{lemma}

\begin{proof}
Note that we have \(a_{kn} \leq k a_n\) by the additivity property.
In particular \(\frac{a_{kn}}{kn} \leq \frac{a_n}{n}\).

Let \(I = \inf \frac{a_n}{n}\).

For any \(\epsilon > 0\) we can find \(N\) with \(\frac{a_N}{N} \leq I + \frac{1}{2}\epsilon\).

For \(n \geq N\), write \(n = kN + r\) with \(0 \leq r < N\).
Then

\begin{align*}
\frac{a_n}{n} &\leq \frac{a_{kN}{n}} + \frac{a_r}{n}\\
&\leq \frac{a_{kN}{kN}} + \frac{N a_1}{n}\\
&\leq I + \frac{1}{2} \epsilon + \frac{N a_1}{n}\\
\end{align*}

So if we pick \(N' = \max(N, \frac{2 N a_1}{\epsilon}) + 1\) the second term is \(< \frac{1}{2}\epsilon\),
and so for \(n \geq N'\) we have \(I \leq \frac{a_n}{n} < \epsilon\).
\end{proof}.

\begin{lemma}{Cesaro's Lemma}
Let \(b_n\) be a monotonic increasing sequence of real numbers such that \(b_n \to \infty\).
Let \(x_n \to x\).
Then \(s_n = \frac{1}{b_n} \sum\limits_{k = 1}^n (b_k - b_{k - 1}) x_n \to x\).
\end{lemma}

\begin{proof}
Let \(\epsilon > 0\).
Pick \(N\) such that whenever \(n \geq N\) we have \(|x_n - x| < \epsilon\).

Let \(e_n = x - x_n\).

Then 

\begin{align*}
s_n &= \frac{1}{b_n} \sum\limits_{k = 1}^n (b_k - b_{k - 1}) x_n\\
&= \frac{b_{N - 1}}{b_n} s_{N - 1} + \frac{1}{b_n} \sum\limits_{k = N}^n (b_k - b_{k - 1}) (x + e_n) \\
&= \frac{b_{N - 1}}{b_n} s_{N - 1} + \frac{1}{b_n} \sum\limits_{k = N}^n (b_k - b_{k - 1}) e_n + \frac{b_n - b_N}{b_n} x \\
\end{align*}

So for \(n \geq N\) we have:

\begin{align*}
|s_n - x| \leq \frac{b_{N - 1}}{b_n} s_{N - 1} + \frac{1}{b_n} \sum\limits_{k = N}^n (b_k - b_{k - 1}) e_n + \frac{b_N}{b_n} x \\
&\leq \frac{b_{N - 1}}{b_n} s_{N - 1} + \frac{1}{b_n} \sum\limits_{k = N}^n (b_k - b_{k - 1}) \epsilon + \frac{b_N}{b_n} x \\
&\leq \frac{b_{N - 1}}{b_n} s_{N - 1} + \frac{b_n - b_N}{b_n} \epsilon + \frac{b_N}{b_n} x \\
&\to \epsilon \\
\end{align*}

Thus by choosing \(N'\) sufficiently large we may ensure \(|s_n - x| \leq \epsilon\) for \(n \geq N'\).
\end{proof}

\begin{lemma}{The Dini Lemma}
Let \(X\) be a compact topological space and \(f_n: X \to \mathbb{R}\) be a pointwise monotonic decreasing sequence of non-negative continuous functions such that \(f_n \to f\) pointwise.
Then \(\sup\limits_{x \in X} f(x) = \lim\limits_{n \to \infty} \sup\limits_{x \in X} f_n(x)\).

Note that continuity of \(f\) is not assumed.
\end{lemma}

\begin{proof}
The sequence \(\sup\limits_{x \in X} f_n(x)\) is monotone decreasing and bounded below by zero,
so it converges to some value \(M\).

Let \(t < M\),
and define \(L_n = \{x: f_n(x) \geq t\}\).
This is a closed subset of \(X\),
is non-empty because \(\sup f_n \geq M > t\),
and because the sequence is monotone decreasing we must have that if \(m < n\),
\(L_n \subseteq L_m\),
so it has the chain conditon and thus non-empty finite intersections.

Thus \(L = \bigcap L_n\) is non-empty,
by compactness of \(X\).
But for any \(x \in L\) we must have \(f(x) \geq t\),
as it is the limit of a set of points \(f_n(x) \geq t\).
Thus \(\sup f(x) \geq t\).
But \(t\) was arbitrary other than being \(< M\),
so we must have \(\sup f(x) \geq M\). 

But \(f(x) \leq f_n(x) \leq \sup f_n(x) \to M\),
so we must have \(\sup f(x) \leq M\),
and thus \(\sup f(x) = M\) as desired.
\end{proof}

\begin{corollary}[Dini's Theorem]
Let \(X\) be a compact topological space,
\(f_n: X \to \mathbb{R}\) a monotone increasing sequence of continuous functions,
and \(f: X \to \mathbb{R}\) a continuous function.
If \(f_n \to f\) pointwise,
then \(f_n \to f\) uniformly.
\end{corollary}

\begin{proof}
Apply the lemma to \(f - f_n\).
\end{proof}

\begin{lemma}[Abel's Theorem]
If \(\sum a_n = l\) then \(sum t^n a_n \to l\) as \(t \to 1^-\).
\end{lemma}

\begin{proof}
Let \(s_n = \sum\limits_{i = 1}^n a_n\), with \(s_0 = 0\).
Let \(f(t) = \sum\limits_{i = 1}^\infty a_i t^i\).

Then
\begin{align*}
f(t) &= \sum\limits_{i = 1}^\infty (x_i - x_{i - 1}) t^n\\
&= \sum\limits_{i = 1}^\infty x_i t^i -  \sum\limits_{i = 1}^\infty x_{i - 1} t^i\\
&= \sum\limits_{i = 1}^\infty x_i t^i -  t \sum\limits_{i = 1}^\infty x_i t^i\\
&= (1 - t) \sum\limits_{i = 1}^{\infty} x_i t^i\\
\end{align*}

Let \(\alpha = \sum a_n\) and \(r_n = \sum\limits_{k > n} a_k\).

Note that \(\sum t^n = {(1 - t)}^{-1}\),
so \(\alpha = (1 - t) \sum \alpha t^n\).
Thus \(\alpha - f(t) = (1 - t) \sum (\alpha - x_n) t^n = (1 - t) \sum r_n t^n \).

Now let \(\epsilon > 0\).

Now pick \(N\) such that for \(n \geq N\),
\(|r_n| < \frac{1}{2}\epsilon\).

Then \(|\alpha - f(t)| \leq (1 - t) \sum\limits_{n = 0}^N |r_n| t^n + (1 - t) \sum\limits_{n = N}^\infty \frac{1}{2}\epsilon t^n \leq (1 - t) M + \frac{1}{2} \epsilon\) for some \(M\).

So for \(t\) sufficently close to \(1\),
\(|f(t) - \alpha| < \epsilon\) as desired.
\end{proof}

\begin{corollary}
Let \(f\) be an analytic function with Taylor series about zero \(f(z) = \sum a_n z^n\),
with radius of convergence \(R\).
If \(|z| = R\) and \(\sum a_n z^n\) converges,
then \(\sum a_n z^n = f(z)\).
\end{corollary}

\begin{proof}
Apply Abel's theorem to \(t \to f(tz)\).
\end{proof}
