\chapter{Statistics}

\section{Estimating Entropy}

For a paper I'm working on I'm interested in the problem of how to estimate the entropy of a distribution.
This appears to be a hard problem.

\cite{DBLP:journals/neco/Paninski03} showed that there is no unbiased non-parametric estimator for the entropy of a discrete distribution.
The proof is straightforward and cute.

\begin{theorem}
Let \(T\) be a multinomial distribution taking values in \(A\).
For a fixed sample size \(N\) there is no unbiased estimator of \(h(T)\).
\end{theorem}

\begin{proof}
Such an estimator would be defined by a function \(q: A^N \to \mathbb{R}\).
Let \(S = (T_1, \ldots, T_N)\) be our sample distribution.
Then \(E(q(S)) = \sum\limits{s \in A^N} q(s) P(S = s)\).
But \(P(S = s)\) is a polynomial in \(A^N\), so \(E(q(S))\) is a polynomial function of the distribution of \(T\).

But the entropy is a transcendental function,
so cannot be equal to \(E(q(S))\) everywhere.
\end{proof}

Nevertheless, 
\cite{1410.5002} demonstrates a neat unbiased estimator for entropy!
The trick is that instead of having a fixed sample size we have an unknown,
possibly unbounded, sample size.

\begin{theorem}
Let \(X\) be some discrete random variable and let \(H(n)\) be the nth harmonic number.
Consider a sequence of IID \(X_1, \ldots, X_n, \ldots\).
Define \(U = \min\limits_{n > 1} X_n = X_1\).
Then \(E(H(U - 1)) = h(X)\),
the entropy of \(X\).
\end{theorem}

\begin{proof}
Let \(X\) take values \(a_i\) and \(P(X = x_i) = p_i\).

Conditioning on \(X_1 = a_i\),
the probability of taking \(n\) more draws to draw \(i\) again is \((1 - p_i)^{n - 1} p_i\),
i.e. it's a geometric distribution.
Thus \(E(H(U) | X_1 = a_i) = p_i \sum H(n - 1) (1 - p_i)^{n - 1}\).
By Proposition~\ref{prop:sumofharmonicpowers},
this sum is \(-\log p_i\).
Taking the expectation of these conditional expectations,
we get the entropy as desired.

\end{proof}

This particular estimator isn't terribly practical,
but they suggest averaging a sequence of them.
If you define \(U_j = \min\limits_{k \geq 1} X_j = X_{j + k}\),
then \(\sum\limits_{i = 1}^n H(U_i - 1)\) is similarly an unbiased estimator,
but with lower variance.

The paper does not provide calculations for these variances.
Perhaps the techniques of \cite{DBLP:journals/dam/FlajoletGT92} would be useful?

Also, a question: Would simulating this process from a bootstrap give an unbiased estimator?

\subsubsection{Interesting Links}

\begin{itemize}
\item \href{http://thirdorderscientist.org/homoclinic-orbit/2013/7/23/analyzing-and-bootstrapping-our-way-to-a-better-entropy-estimator-mdash-part-ii-computational}{Analyzing and Bootstrapping our Way to a Better Entropy Estimator}.
\item \href{https://math.stackexchange.com/questions/604654/estimating-the-entropy}{Good stack exchange discussion on this}
\item \href{https://memming.wordpress.com/2014/02/09/a-guide-to-discrete-entropy-estimators/}{A guide to discrete entropy estimators}
\end{itemize}
