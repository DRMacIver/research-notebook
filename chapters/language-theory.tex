\chapter{Misc Formal Language Theory}

\section{Regular Relationships}

Note: this sections somewhat under construction.
There's an intuitive notion I'm trying to capture that I haven't quite got to yet,
do do with test-case reduction and notions of non-deterministic transformations between unknown formal languages,
but I haven't got there yet.

These ideas \emph{must} correspond to something well known,
but if so I haven't been able to find it.

\begin{definition}
Let \(A\) be some alphabet. The set of regular relationships on \(A^*\) is the smallest set \(\mathcal{G} \subseteq (A^*)^2\) such that:

\begin{itemize}
\item \(\emptyset \in \mathcal{G}\)
\item \( (A^*)^2 \in \mathcal{G}\)
\item \(\{(x, y): x = y\} \in \mathcal{G}\)
\item \(\{(x, y): y \in A^*\}\) for each \(x \in A^*\).
\end{itemize}

Which is closed under the following operations:

\begin{itemize}
\item \(R \to R^c\)
\item \(R \to \{(v, u): (u, v) \in R\)
\item \(R_1, R_2 \to R_1 \cap R_2\)
\item \(R_1, R_2 \to \{(ru, sv) (r, s) \in R_1, (u, v) \in R_2\}\)
\item \(R_1, R_2 \to \{(u, v): \exists s, (u, s) \in R_1 \wedge (s, v) \in R_2 \} \)
\item \(R \to R^+ = \{(u, v_1 \ldots v_n) (u, v_i) \in R\}\)
\end{itemize}

\end{definition}

\begin{conjecture}
If \(R\) is a regular relationship and \(L\) is a regular language,
then \(\{u: \exists v \in L, (u, v) \in R\}\) is also a regular language.
\end{conjecture}

\begin{proof}
TODO: This should just be a matter of checking closure under the operations.
\end{proof}

Note that if this conjecture is true then the transitive closure of a regular relationship need not be regular.
Let \(A = \{0, 1\}\).
The relationship \(y = u01v\) and \(x = uv\) for some \(u, v\) is easily seen to be regular.
However when \(x = \epsilon\),
the set of suitable \(y\) is a bracket balancing language,
which is a classic example of a context free but not regular language.

This definition isn't super easy to work with (and is possibly not quite right),
so here's another definition:

\begin{definition}
Let \(A\) be some alphabet and let \(\phi \not\in A\).
Let \(C = (A \cup \{\phi\})^2 - \{(\phi, \phi)\}\).

A relationship \(\sim \subseteq (A^*)^2\) is \emph{jump-regular} if there is some regular language \(\mathcal{L} \subseteq C^*\) such that \(x \sim y\) if and only if \(\mathcal{L} \cap \mathcal{J}(x, y) \neq \emptyset\),
where \(\mathcal{J}(x, y)\) is the \emph{jump language} of \(x\) and \(y\) defined as:

\begin{enumerate}
\item \(\mathcal{J}(x, \epsilon) = [(x_i, \phi) | i]\)
\item \(\mathcal{J}(\epsilon, y) = [(\phi, y_i) | i]\)
\item \(\mathcal{J}(au, bv) = \mathcal{J}(u, v) \cup \mathcal{J}(au, v) \cup \mathcal{J}(u, bv)\)
\end{enumerate}

i.e. the jump language is the language defined by an automaton that at each stage consumes at least one character from each of its arguments,
and any character that does not come from its arguments is \(\phi\), the jump character,
which accepts all walks which consume both strings fully.

\end{definition}


\begin{conjecture}
Every jump-regular language is regular.
\end{conjecture}

It is unclear to me that jump-regular predicates are closed under the boolean operations,
so I'm not sure if there are regular but non jump-regular relationships.

There's a particularly easy class of jump-regular relationships:

\begin{definition}
For \(x, y \in A^*\), define the zip of \(x\) and \(y\) as \(\zeta(x, y) = [(\iota(x, i), \iota(y, i)) | 0 \leq i < \min(|x|, |y|)]\).
i.e. we \(\phi\)-extend the shorter list so that the two are of the same length and then pair up all characters at the same index.
\end{definition}

\begin{lemma}
If \(\sim\)
is equivalent to
\(\zeta(x, y) \in \mathcal{L}\)
for some regular language \(\mathcal{L}\),
then \(\sim\) is regular. 
\end{lemma}

\begin{proof}
Let \(Z = (A^2)^* (A \times \{\phi\} \cup \{\phi\} \times A)^*\).
Then \(t \in Z\) if and only if \(t = \zeta(x, y)\) for some \(x, y\),
and \(Z\) is clearly regular.

Further, clearly \(\zeta(x, y) \in \mathcal{J}(x, y)\).

Thus \(\mathcal{L}' = \mathcal{L} \cap Z\) is a regular language such that \(x \sim y\) if and only if \(\mathcal{J}(x, y) \cap \mathcal{L}' \neq \emptyset\),
so \(\mathcal{\sim}\) is regular as desired.

\end{proof}

\begin{proposition}
The relationship \(|x| < |y|\) is regular.
\end{proposition}

\begin{proof}
It corresponds to \(\zeta(x, y) \in C^* (\phi, \cdot)\).
\end{proof}

\begin{corollary}
The relationship \(|x| = |y|\) is regular.
\end{corollary}

\begin{proof}
\(|x| = |y| \iff \neg (|x| < |y| \vee |y| < |x|)\).

(this is also easy to prove directly).
\end{proof}

\begin{proposition}
The relationship \(x \leq y\) lexicographically is regular.
\end{proposition}

\begin{proof}
It's easy enough to draw a three-state DFA for this:

\begin{enumerate}
\item Initial state. Accepting. For \((x, y) | x = y\), transitions to self.
For \((x, \phi)\), transition to state \(3\). For \((\phi, x)\) transition to state 2.
For \(x < y\), transition to \(2\). For \(x > y\), transition to 3.
\item Accepting state, unconditionally transitions back to itself.
\item Non-accepting state, unconditionally transitions back to itself.
\end{enumerate}
\end{proof}


\begin{theorem}
If \(\sim\) is a jump-regular relationship then and \(L \subseteq A^*\) is a regular language then \(\{x: \exists y \in L, x \sim y\}\) is regular.
\end{theorem}

\begin{proof}
Let \(X\) be an NDFA for \(\sim\) and \(Y\) and NDFA for \(L\).
For convenience assume neither has \(\epsilon\)-transitions.

Build an NDFA on \(S_X \times S_X\) as follows:
The starting point is \((x_0, y_0)\) and a state is accepting if and only if both components are.

\begin{enumerate}
\item For each \(s \in S_X\) and \(a \in A\), add an \(\epsilon\)-transition from \((s, t)\) to \((\delta_X(s, (\phi, a)), \delta_Y(t, a))\)
\item For each \(s \in S_X\) and \(a \in A\), add an \(a\)-transition from \((s, t)\) to \((\delta_X(s, (a, \phi)), t)\)
\item For each \(s \in S_X\) and \(a, b \in A\), add an \(a\)-transition from \((s, t)\) to \((\delta_X(s, (a, b)), \delta_Y(t, b))\)
\end{enumerate}

Accepting paths through this state machine should correspond precisely to members of the jump language for some \(y \in L\).

TODO: Check details.
\end{proof}

