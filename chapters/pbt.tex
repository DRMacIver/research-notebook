\chapter{Property Based Testing Literature Review}

This is mostly an index of existing property-based testing related papers,
with some commentary.

The foundational paper is of course ``QuickCheck: a lightweight tool for random testing of Haskell programs''\cite{DBLP:conf/icfp/ClaessenH00}.
The core idea here is that you can express testing as if it were formal verification,
by using the type signature of a function to fill in random values.

I am not a fan of this idea,
and think that type driven property based testing is a historical accident that has largely served as an attractive nuisance that has prevented people from implementing better solutions.

``Smallcheck and lazy smallcheck: automatic exhaustive testing for small values''\cite{DBLP:conf/haskell/RuncimanNL08} is in some sense the other foundational paper in this area
but seems to have proven less popular (and I'm suspicious about its efficacy),
based on bounded exhaustive enumeration.

``EasyCheck---Test Data for Free''\cite{DBLP:conf/flops/ChristiansenF08} is an interesting extension where the generation is based on randomized tree enumeration,
where the trees are expressed with non-determinism operators (or just ADTs).

``Type Targeted Testing''\cite{DBLP:conf/esop/SeidelVJ15} is based on using refinement types to satisfy tricky preconditions.

``Targeted property-based testing``\cite{DBLP:conf/issta/LoscherS17} is about some specialisations for the case where the property depends on some numeric value.
Suppose you have some property \texttt{assert f(x) <= bound}.
This will tend to be falsified early or not at all---purely
random testing explores its boundary values very quickly (the probability of the maximum occurring at \(n\) tests is \(\frac{1}{n}\)).
But by doing a simple hill climbing algorithm based on mutating found values to increase their score,
this can very easily be found.

``Find more bugs with QuickCheck!''\cite{DBLP:conf/icse/HughesNSA16} is about the problem where certain bugs tend to dominate:
If you have one very likely bug,
it will tend to dominate your testing and distract you from finding other bugs.
In this paper (which is specifically about Erlang QuickCheck's model based testing) they propose using sequences of method calls found in a bug to mark examples to avoid when generating later.

``A PropEr integration of types and function specifications with property-based testing''\cite{DBLP:conf/erlang/PapadakisS11} is about PropEr,
the other Erlang property-based testing library.
I haven't actually read it yet.

In the other direction, ``The New Quickcheck for Isabelle''\cite{DBLP:conf/cpp/Bulwahn12} is about a QuickCheck for Isabelle,
which is a theorem prover

``FitSpec: refining property sets for functional testing''\cite{DBLP:conf/haskell/BraquehaisR16} is about using mutation of tested functions to see whether a set of properties is minimal and complete.

\section{Reading List}

Here are some interesting citing papers that might be worth reading:

\begin{itemize}
\item ``Beginner's luck: a language for property-based generators''\cite{DBLP:conf/popl/LampropoulosGHH17}
\item Automating Black-Box Property Based Testing\cite{DBLP:phd/basesearch/Duregard16}
\end{itemize}

\section{History and Etymology}

These days ``property-based testing'' seems to be used almost synonymously with ``QuickCheck derived tooling'',
but this turns out to be somewhat of a historical anomaly.

The earliest usage of the term that I can find is in the work of George Fink,
leading up to his PhD thesis ``Discovering security and safety flaws using property-based testing''\cite{fink1996discovering}.
The paper which introduces the term is ``Towards a property-based testing environment with applications to security-critical software''\cite{fink1994towards}.

This sense of ``property-based testing'' appears to be mostly unrelated to the modern sense:
The properties are expressed in TASPEC,
a language which as far as I can tell is to a derivative of Z,
and the implementation is based on program slicing.

The original QuickCheck paper\cite{DBLP:conf/icfp/ClaessenH00} never uses the term ``property-based testing'' (``property-based specifications'' appears once in the paper).
The first time it is used in a QuickCheck paper is in ``Testing telecoms software with quviq QuickCheck''\cite{DBLP:conf/erlang/ArtsHJW06},
where it is not explicitly introduced they just say:

\begin{quote}
Quviq QuickCheck is a property-based testing tool, developed from
Claessen and Hughes' earlier QuickCheck tool for Haskell
\end{quote}

The timeline for this is that the first paper was published in 2000,
with the telecoms one being published in 2006.
2006 was also the year that Quviq was founded (May 2006 according to the archive.org version of their site from then).
The term ``property-based testing'' is not used in their initial flyer,
soo that gives a fairly narrow window for when they started using it---some
time in 2006.

Based on analysis of the mailing lists and IRC archives for the Haskell community,
the term then got backported from Erlang to Haskell.
The words ``property based'' don't really appear together in the mailing lists (haskell or haskell-cafe) or IRC until 2007.

There is however some use of the term ``property-based testing'' in the intervening years in a way that seems to align with the modern usage.
For example ``A Metamorphic Approach to Integration Testing of Context-Sensitive Middleware-Based Applications''\cite{DBLP:conf/qsic/ChanCL05} refers to metamorphic testing as ``a property-based testing strategy'' in 2005.

I have emailed John Hughes to ask him if he remembers when and why they started using the term.
