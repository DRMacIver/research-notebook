\chapter{Some Theory of Test-Case Reduction}

This chapter is mostly for annoyingly pedantic theory of test-case reduction that nobody except me cares about.

\newcommand{\exemplars}[1]{\mathcal{E}(#1)}

Let \(T\) be some non-empty set.
Define \(\exemplars{T} = \{ (U, x) : U \in \pow(T), x \in U \}\).

\begin{definition}
A test-case reducer is a function \(r: \exemplars{T} \to T\) such that \(r(U, x) \in U\) and \(r(U, r(U, x)) = r(U, x)\)
\end{definition}

\begin{definition}
A test-case reducer \(r\) normalizes \(U \in \pow(T)\) if for all \(x, y \in U\), \(r(U, x) = r(U, y)\).
If \(r\) normalizes all \(U \in \pow(T)\) is is said to be normalizing.
\end{definition}


\newcommand{\preckap}{\stackrel{\kappa}{\prec}}

\begin{definition}
Let \(\kappa: A \to \pow(A)\).
Define \(x \preckap y\) if \(x \in \kappa(y)\).

A reducer \(r\) is \(\kappa\)-minimizing if \(\kappa(r(U, x))\) is always an \(\preckap\)-minimal element of \(U\).
It is \(\kappa\)-local if it is \(\kappa\)-minimizing and for any \(x \in U\) there is some sequence \(r(U, x) = s_0 \preckap s_1 \preckap \ldots \preckap s_n = x)\) with \(s_{i + 1} \in U \cap \kappa(s_i)\).
\end{definition}

Most generalised delta-debugging is \(\kappa\)-minimizing for some \(\kappa\),
and most greedy implementations are \(\kappa\)-local.

\begin{proposition}
There exists a \(\kappa\)-minimizing reducer if and only if \(\preckap\) is well-founded.
\end{proposition}

\begin{proof}
If \(U \subseteq T\) is non-empty and \(r\) is a \(\kappa\)-minimizing reducer,
then \(r(U, x)\) for any \(x \in U\) is a \(\preckap\)-minimal element of \(U\),
hence \(\preckap\) is well-founded.

Suppose \(\preckap\) is well-founded,
then an \(r\) which just picks any \(\preckap\)-minimal element of \(U\) is a \(\kappa\)-minimizing reducer.
\end{proof}

\begin{lemma}
A well-founded relation is antisymmetric and irreflexive.
\end{lemma}

\begin{proof}
If we have \(x \prec x\) then \(\{x\}\) has no \(\prec\)-minimal element,
so \(\prec\) is irreflexive.

Consider \(x \neq y\).
\(\prec\) is total,
so without loss of generality assume \(x \prec y\).
If \(y \prec x\),
then \(\{x, y\}\) has no minimal element,
contradicting well-foundedness.
Hence \(\prec\) is antisymmetric.
\end{proof}

\begin{lemma}
Let \(\prec\) be a well-founded total relation,
then \(\prec\) is a strict well order.
\end{lemma}

\begin{proof}
If \(x \prec y\),
\(y \prec x\)
and \(y \prec z\) then \(\{x, y, z\}\) has no minimal element.
Therefore \(z \not\prec x\),
and so by totality we must have \(x \prec z\).
Hence \(\prec\) is transitive.

We already know \(\prec\) is well-founded and total by assumption,
so by the previous lemma it is also antisymmetric and irreflexive,
hence it is a strict well order.
\end{proof}

\begin{theorem}
A \(\kappa\)-local reducer is normalizing if and only if \(\preckap\) is a strict well order.

In this case there exists a unique \(\kappa\)-local reducer defined by \(r(U, x) = \min\limits_{\preckap} U\).
\end{theorem}

\begin{proof}
Let \(r\) be a normalizing \(\kappa\)-local reducer.

Suppose \(x \neq y\).
Then by the fact that \(r\) is normalizing,
we must have \(r(\{x, y\}, x) = r(\{x, y\}, y)\).
Suppose without loss of generality that \(r(\{x, y\}, y) = x\).
Then because \(r\) is \(\kappa\)-local we must have \(x \preckap y\).
Hence \(\preckap\) is total.
Thus by the preceding lemma it is a strict total order.

The uniqueness comes from the fact that a total order can only have a single minimal element in a set,
so necessarily we have \(r(U, x) = \min\limits_{\preckap} U\).
The existence comes from the fact that \(r\) defined thus is a \(\kappa\)-local reducer.
\end{proof}

Another interesting way to force this condition is the following.

\begin{definition}
A reducer \(r\) is constraint consistent if \(r(U, x) = r(U', x)\) for any \(U'\) with \(\{x, r(U, x)\} \subseteq U' \subseteq U\).
\end{definition}

\begin{theorem}
If \(r\) is constraint-consistent then there exists a unique \(\kappa\) such that \(r\) is \(\kappa\)-local,
and \(\preckap\) is a strict partial order.
\end{theorem}

\begin{proof}
Define \(\kappa(x) = \{y \neq x: r(\{x, y\}, x) = y\}\).

\(\preckap\) is anti-symmetric,
as if \(r(\{x, y\}, y) = x\) then (because \(r\) is a reducer),
\(r(\{x, y\}, x) = x \neq y\).

Taking \(U' = \{r(U, x), x\} \subseteq U\),
by constraint-consistency we must have \(r(U', x) = r(U, x)\),
hence \(r(U, x) \in \kappa(x)\) for all \(x \in U\).

Suppose \(y \in U \cap \kappa(r(U, x))\).
Then \(r(U, x) = r(U, r(U, x)) = r(\{r(U, x), y\}, r(U, x)) = y\),
which contradicts the fact that \(x \not \kappa(x)\).
Thus \(r(U, x)\) is \(\kappa\)-minimal,
and \(r\) is \(\kappa\)-minimizing.
The sequence condition is immediate from \(r(U, x) \in \kappa(x)\),
so \(r\) is \(\kappa\)-local.

We now need to show that \(\preckap\) is a strict partial order.
This just requires showing transitivity.
Let \(x \preckap y\) and \(y \preckap z\).
Let \(U = \{x, y, z\}\).
Then \(r(U, z) = x\), because \(x\) is the only \(\kappa\)-minimal elemnt of \(U\).
By constraint-consistency,
\(r(\{x, z\}, z) = x\),
hence \(x \preckap z\) as desired.
\end{proof}

In practice what all of these results mean is that getting consistent behaviour out of test case reducers is bloody hard---normalization
requires bounded exhaustive enumeration,
and even constraint consistency requires enumerating the transitive closure of your smallest relationship.
e.g. a constraint consistent attempt to do one-minimality would actually have to enumerate every subset.
