\documentclass[a4paper,oneside]{book}

%% Language and font encodings
\usepackage[english]{babel}
\usepackage[utf8x]{inputenc}
\usepackage[T1]{fontenc}

%% Sets page size and margins
\usepackage[a4paper,top=3cm,bottom=2cm,left=3cm,right=3cm,marginparwidth=1.75cm]{geometry}

%% Useful packages
\usepackage{amsmath}
\usepackage{amssymb}
\usepackage{amsthm}
\usepackage{graphicx}
\usepackage[colorinlistoftodos]{todonotes}
\usepackage[colorlinks=true, allcolors=blue]{hyperref}
\usepackage{listings}
\usepackage{mathtools}

\usepackage{etoolbox}
\usepackage{setspace}
\AtBeginEnvironment{quote}{\singlespacing\small}

\setlength{\parskip}{0.5em}

\usepackage{mathtools}
\DeclarePairedDelimiter{\ceil}{\lceil}{\rceil}
\DeclarePairedDelimiter{\floor}{\lfloor}{\rfloor}

\newtheorem{definition}{Definition}[section]
\newtheorem{proposition}{Proposition}[section]
\newtheorem{lemma}{Lemma}[section]
\newtheorem{corollary}{Corollary}[section]
\newtheorem{theorem}{Theorem}[section]
\newtheorem{example}{Example}[section]
\newtheorem{claim}{Claim}[section]

\newcommand{\pow}[1]{\mathcal{P}(#1)}

% lstlisting macros
\lstset{
  language=Python, % choose the language of the code
  showspaces=false, % show spaces adding particular underscores
  showstringspaces=false, % underline spaces within strings
  showtabs=false, % show tabs within strings adding particular underscores
  tabsize=8, % sets default tabsize to 2 spaces
  captionpos=b, % sets the caption-position to bottom
  mathescape=true, % activates special behaviour of the dollar sign
  basicstyle=\footnotesize\tt,
  columns=fullfelxible,
  xleftmargin=2em,
  %commentstyle=\rmfamily\itshape,
  morekeywords={},
  escapeinside={(*@}{@*)},
  numbers=left
}


\title{Literature Review for Hypothesis PhD}
\author{David R. MacIver}

\begin{document}

\maketitle

\tableofcontents

\chapter{Introduction}

This is my literature review for my PhD.
It's very much a work in progress.

In it's present form it is currently intended as a skeleton around which I will build things,
and does not contain much text that I expect will end up in the final ``product''.

\chapter{Outline}

Questions I would like to provide partial answers to:

\begin{itemize}
\item What is property-based testing?
\item What is the proper context for property-based testing?
\item What is the significance of Hypothesis?
\item What are some interesting research directions for improving Hypothesis?
\end{itemize}

\chapter{Unincorporated Material}

This chapter gathers papers that I must/should/could include in my review,
along with some notes on them,
but does not attempt to do any significant amount of synthesis.

\section{Must Haves}

\subsection{QuickCheck Papers}

The foundational paper for my work is of course ``QuickCheck: a lightweight tool for random testing of Haskell programs''\cite{DBLP:conf/icfp/ClaessenH00}.

There are a number of other papers worth referencing:

\begin{itemize}
\item ``Testing telecoms software with quviq QuickCheck''\cite{DBLP:conf/erlang/ArtsHJW06}
\item ``SmartCheck: automatic and efficient counterexample reduction and generalization''\cite{DBLP:conf/haskell/Pike14} is one of the few papers about improving QuickCheck's test case reduction,
so definitely worth including.
\end{itemize}

\subsection{Test Case Reduction Papers}

\begin{itemize}
\item ``Simplifying failure-inducing input''\cite{DBLP:conf/issta/HildebrandtZ00} is the delta-debugging paper that you have to cite if you ever write anything about test-case reduction.
\item ``One test to rule them all''\cite{DBLP:conf/issta/GroceHK17} is particularly relevant to Hypothesis because of the relationship between normalization and its shortlex-minimization goal.
\end{itemize}

\subsection{Misc}

\begin{itemize}
\item ``Minimization of Randomized Unit Test Cases''\cite{DBLP:conf/issre/LeiA05} is a good justification for the combination of random test case generation and test-case reduction.
\end{itemize}

\begin{itemize}
\item ``Simplifying failure-inducing input''\cite{DBLP:conf/issta/HildebrandtZ00} is the delta-debugging paper that you have to cite if you ever write anything about test-case reduction.
\item ``One test to rule them all''\cite{DBLP:conf/issta/GroceHK17} is particularly relevant to Hypothesis because of the relationship between normalization and its shortlex-minimization goal.
\end{itemize}

\section{Should Haves}

\begin{itemize}
\item ``Targeted property-based testing''\cite{DBLP:conf/issta/LoscherS17} is about how you can do what is basically hill climbing with restart to better test properties.
This is relevant mostly as an example of why it's nice to be able to extend with new operations on generated data.
\item ``Why is random testing effective for partition tolerance bugs?''\cite{DBLP:journals/pacmpl/MajumdarN18} is an interesting argument that random testing can and should work.
\item ``Partition Testing Does Not Inspire Confidence``\cite{DBLP:journals/tse/HamletT90} is a good argument that random testing is pretty OK.
\item ``Behind Human Error''\cite{BehindHumanError} contains a lot of lucid discussion about error that I think is valuable for this sort of work.
\item ``An Experimental Evaluation of the Assumption of Independence in Multiversion Programming''\cite{DBLP:journals/tse/KnightL86}---a
lot of property-based testing is basically multiversion programming to do differential testing.
\end{itemize}

\section{Could Haves}

\begin{itemize}
\item ``Can a Machine Design?''\cite{doi:10.1162/07479360152681083} contains a really nice account of different modes of human/computer codesign,
and how having the computer make things for a human to correct is much less stressful than having the computer correct the human.
\item ``When and what to automate in software testing? {A} multi-vocal literature review''\cite{DBLP:journals/infsof/GarousiM16}---this
seems a natural fit.
\end{itemize}

\input{./includes/bibsection.tex}

\end{document}
