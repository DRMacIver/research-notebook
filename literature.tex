\documentclass[a4paper,oneside]{book}

%% Language and font encodings
\usepackage[english]{babel}
\usepackage[utf8x]{inputenc}
\usepackage[T1]{fontenc}

%% Sets page size and margins
\usepackage[a4paper,top=3cm,bottom=2cm,left=3cm,right=3cm,marginparwidth=1.75cm]{geometry}

%% Useful packages
\usepackage{amsmath}
\usepackage{amssymb}
\usepackage{amsthm}
\usepackage{graphicx}
\usepackage[colorinlistoftodos]{todonotes}
\usepackage[colorlinks=true, allcolors=blue]{hyperref}
\usepackage{listings}


\usepackage{mathtools}
\DeclarePairedDelimiter{\ceil}{\lceil}{\rceil}
\DeclarePairedDelimiter{\floor}{\lfloor}{\rfloor}

\newtheorem{definition}{Definition}[section]
\newtheorem{proposition}{Proposition}[section]
\newtheorem{lemma}{Lemma}[section]
\newtheorem{corollary}{Corollary}[section]
\newtheorem{theorem}{Theorem}[section]
\newtheorem{example}{Example}[section]
\newtheorem{claim}{Claim}[section]


% lstlisting macros
\lstset{
  language=Python, % choose the language of the code
  showspaces=false, % show spaces adding particular underscores
  showstringspaces=false, % underline spaces within strings
  showtabs=false, % show tabs within strings adding particular underscores
  tabsize=8, % sets default tabsize to 2 spaces
  captionpos=b, % sets the caption-position to bottom
  mathescape=true, % activates special behaviour of the dollar sign
  basicstyle=\footnotesize\tt,
  columns=fullfelxible,
  xleftmargin=2em,
  %commentstyle=\rmfamily\itshape,
  morekeywords={},
  escapeinside={(*@}{@*)},
  numbers=left
}


\title{Literature Review for Hypothesis PhD}
\author{David R. MacIver}

\begin{document}

\maketitle

\tableofcontents

\chapter{Where are we going and what are we doing in this handbasket?}

NB:\ This literature review is very much a work in progress and is not really ready for public consumption.
It is and will be highly fragmented,
with large missing gaps and much of the bits that are here only in a temporary state that will later be reworked into other things.
Proceed with caution.

How to explain? How to describe? Even the omniscient viewpoint quails\cite{0812515285}.

Let me begin with some context.

The focus of my PhD is an open source project I wrote (and many others now also contribute to) called \emph{Hypothesis}.
Hypothesis was originally purely a Python library\footnote{\url{https://github.com/HypothesisWorks/hypothesis-python}},
but now also has a version for Ruby\footnote{\url{https://github.com/HypothesisWorks/hypothesis-ruby}}.
I will blur the distinction and simply refer to them all as ``Hypothesis''.

Hypothesis is a library in the spirit of QuickCheck\cite{DBLP:conf/icfp/ClaessenH00} for \emph{property-based testing}.
I will discuss in more detail what property-based testing actually \emph{is} in TODO,
but the distinguishing feature of both Hypothesis and QuickCheck is that they provide generated data for user-defined test functions,
allowing users to write tests that quantify over a range of data---instead
of testing with a specific string,
they can write a test that logically tests \emph{all} strings (although in practice they only test a finite number of strings).

Hypothesis is an interesting vehicle for research for two main reasons:

\begin{itemize}
\item It is arguably the first widely used tool of its type that does not target a functional programming language\footnote{
The other contender is theft (\url{https://github.com/silentbicycle/theft/}).
Comparisons are odious,
and theft is an excellent project,
but Hypothesis almost certainly has significantly more usage than theft,
and probably has for most of their shared lifetime.
},
and thus has many different challenges and aspects to it that are not present in the existing literature on property-based testing.
\item Hypothesis has a novel and unusual underlying model,
in which random generation is reframed as a parsing operation.
This leads to a number of improvements,
both current and potential,
in its capabilities compared to existing property-based testing implementations.
I will discuss these further in TODO.\ 
\end{itemize}

Being based first and foremost on the work and impact in a software testing tool that is used in practice,
my research necessarilly comes with a subgoal of bridging academic and industry notions of software testing.
This is already to some extent implicit in the field---as
I will discuss in TODO,
QuickCheck style testing has a fairly large presence in industry.
As a result,
this literature review is necessarily \emph{multivocal} in the sense of~\cite{DBLP:conf/ease/GarousiFM16}---that is,
I will make extensive reference to the so-called ``grey'' literature,
is published outside of the formal peer review system.
I will attempt to use this to explore the following two questions:

\begin{itemize}
\item What is the proper context of property-based testing?
\item How can be improve not just the state of the art in testing research,
but also the adoption of the tools we develop?
\end{itemize}

Much of the latter question falls outside of research per se,
but I believe it is vital to consider it both to increase the impact of research and also because it can inform interesting research questions.

The rest of this literature review follows the following structure:

\begin{enumerate}
\item In Chapter~\ref{chap:quickcheck} I will discuss what QuickCheck is and does,
and explore some of the differences between it and its descendants,
as well as other software testing tools.
\item In Chapter~\ref{chap:purposesoftesting} I will ask ``What is testing \emph{for}?'',
and argue that academia and industry have very different answers to this question.
Within this framework, I will discuss the sub-question ``What is property-based testing and how can it help?''.
\item In Chapter~\ref{chap:testinglanguage} I will talk about some of the novel design choices in Hypothesis,
and discuss active areas of research that will inform the direction of my PhD.
\item Finally in Chapter~\ref{chap:conclusion} I will round off with some closing remarks about the general space of software testing research and its role in bridging academia and industry.
\end{enumerate}

\chapter{Why is everything terrible?}

This chapter isn't written yet.
In this I will bemoan the state of the world and generally talk about how we're all doomed and nothing can fix that.

As James Mickens puts it\cite{mickens2014saddest}:

\begin{quote}
``How can you make a reliable computer service?'' the presenter will ask in an innocent voice before continuing,
``It may be difficult if you can’t trust anything and the entire concept of happiness is a lie designed by unseen overlords of endless deceptive power.''
The presenter never explicitly says that last part,
but everybody understands what's happening.
\end{quote}

\chapter{QuickCheck's friends and family}\label{chap:quickcheck}

QuickCheck was introduced back in 2000,
in ``QuickCheck: a lightweight tool for random testing of Haskell programs''\cite{DBLP:conf/icfp/ClaessenH00}.
It has since been ported to many languages\footnote{\url{https://hypothesis.works/articles/quickcheck-in-every-language/}},
with the most notable one being the commercial Erlang implementation\cite{DBLP:conf/erlang/ArtsHJW06}.

A test in QuickCheck looks something like the following:

\begin{lstlisting}[language=Haskell]
import Test.QuickCheck (quickCheck, (==>), Property)
import Data.List (elem, delete)

prop_remove :: [Int] -> Int -> Property
prop_remove ls x = elem x ls ==> not (elem x (delete x ls))

main = quickCheck prop_remove
\end{lstlisting}

In this test we test the property that given a list of integers and an integer,
with the integer contained in the list,
after removing the integer from the list it is not longer present.

A similar property expressed in Hypothesis would be:

\begin{lstlisting}[language=Python]
from hypothesis import given, assume
import hypothesis.strategies as st


@given(st.lists(st.integers()), st.integers())
def test_deleting(ls, i):
    assume(i in ls)
    ls.remove(i)
    assert i not in ls
\end{lstlisting}

When run, these tests both do the same thing:
They randomly generate a list of integers and an integer,
if the integer is contained in the list they continue the test,
otherwise they mark it as invalid.
Once they have found a sufficient number of valid test cases that did not fail (100 by default) the test stops.
If prior to that point they have found a counter-example to the property,
they begin a process of test-case reduction to attempt to produce a smaller, simpler, example.

The development of QuickCheck and its descendants in the context of Haskell has recently been very ably summarised in Rudy Braquehais's PhD thesis\cite{matela2017tools},
so I do not propose to go into QuickCheck in great detail,
but instead to situate it in the broader context of the software testing literature.

As a result I will take the liberty to be significantly more opinionated in my review of the subject.

The first piece of context to consider is that this combination of random testing and test-case reduction is at this point,
from a research point of view,
\emph{incredibly boring}.
Random testing and test case reduction are both very well studied subjects---in
the latter only since the original QuickCheck paper was published
(the original paper on the subject being of a comparable age to QuickCheck itself\cite{DBLP:conf/issta/HildebrandtZ00}),
but even at the time of publication Random testing was a very well known implementation technique (TODO:\ Citations).

The novel feature of QuickCheck was in fact nothing to do with the quality of its testing,
but that it provided a way of composing these generators together to make it easier to write arbitrary tests over arbitrary data types.

This is not intended as a slight on the authors of QuickCheck,
as they acknowledge this in the paper itself!

\begin{quote}
We have taken two relatively old ideas, namely specifications as oracles and random testing,
and found ways to make them easily available to Haskell programmers.
\end{quote}

Thus from the very beginning the basic question of QuickCheck's design has not been one of advancing the state of the art of software testing research,
but fundamentally a question of usability:
How can we make it easier for people to use these ideas?

Given this,
what happened next may surprise you:
The intervening period has consisted of users of QuickCheck and its descendants holding on to the rather bizarre belief that this sort of testing is in some way tied to functional programming.

For example,
consider the following quote from~\cite{matela2017tools}:

\begin{quote}
Although still useful,
property-based testing is a bit less useful in the realm of imperative programming languages as property-based testing
benefits from testing functions and modules that have no side-effects
\end{quote}

This is not to pick on the author of this quote specifically:
It is a notion I've personally heard many times,
this merely happens to be the most recent.

In one sense this claim is true:
It is much easier to test code that has no side-effects.
On the other hand,
it seems to ignore the fact that testing remains the most common tool for software correctness and is widespread in industry in almost every programming language,
but if anything is \emph{especially} common in dynamically typed imperative programming languages,
which according to this line of reasoning are the hardest to test.

\chapter{Teach me your human notion of software testing}\label{chap:purposesoftesting}

This chapter isn't written yet.
In this chapter I will argue that our notion of what testing is for is flawed:
Most testing cannot be about finding bugs,
because the overwhelming majority of software has no non-trivial bugs until a negotiation involving both the users and the authors of the software concludes that a particular behaviour is a bug.

\chapter{Speaking a test's language}\label{chap:testinglanguage}

This chapter isn't written yet.
This will be a little bit about explaining the implementation of Hypothesis,
and a lot about how this connects up to ideas from fuzzing, language inference, combinatorics (e.g. Boltzmann Samplers), and maybe search-based software testing.
It will also contain a test-case reduction subsection.

\chapter{A new and terrifying age of high quality software}\label{chap:conclusion}

This chapter isn't written yet.
It's basically a more entertaining name for ``Conclusion'',
with reference to the manifesto in the Hypothesis documentation.

\chapter{Unincorporated Material}

This chapter gathers papers that I must/should/could include in my review,
along with some notes on them,
but does not attempt to do any significant amount of synthesis.

\section{Must Haves}

\subsection{QuickCheck Papers}

The foundational paper for my work is of course ``QuickCheck: a lightweight tool for random testing of Haskell programs''\cite{DBLP:conf/icfp/ClaessenH00}.

There are a number of other papers worth referencing:

\begin{itemize}
\item ``Testing telecoms software with quviq QuickCheck''\cite{DBLP:conf/erlang/ArtsHJW06}
\item ``SmartCheck: automatic and efficient counterexample reduction and generalization''\cite{DBLP:conf/haskell/Pike14} is one of the few papers about improving QuickCheck's test case reduction,
so definitely worth including.
\end{itemize}

\subsection{Test Case Reduction Papers}

\begin{itemize}
\item ``Simplifying failure-inducing input''\cite{DBLP:conf/issta/HildebrandtZ00} is the delta-debugging paper that you have to cite if you ever write anything about test-case reduction.
\item ``One test to rule them all''\cite{DBLP:conf/issta/GroceHK17} is particularly relevant to Hypothesis because of the relationship between normalization and its shortlex-minimization goal.
\item ``Minimization of Randomized Unit Test Cases''\cite{DBLP:conf/issre/LeiA05} is a good justification for the combination of random test case generation and test-case reduction.
\end{itemize}

\section{Should Haves}

\begin{itemize}
\item ``Targeted property-based testing''\cite{DBLP:conf/issta/LoscherS17} is about how you can do what is basically hill climbing with restart to better test properties.
This is relevant mostly as an example of why it's nice to be able to extend with new operations on generated data.
\item ``Why is random testing effective for partition tolerance bugs?''\cite{DBLP:journals/pacmpl/MajumdarN18} is an interesting argument that random testing can and should work.
\item ``Partition Testing Does Not Inspire Confidence``\cite{DBLP:journals/tse/HamletT90} is a good argument that random testing is pretty OK.
\item ``Behind Human Error''\cite{BehindHumanError} contains a lot of lucid discussion about error that I think is valuable for this sort of work.
\item ``An Experimental Evaluation of the Assumption of Independence in Multiversion Programming''\cite{DBLP:journals/tse/KnightL86}---a
lot of property-based testing is basically multiversion programming to do differential testing.
\item ``Software Testing Research: Achievements, Challenges, Dreams''\cite{DBLP:conf/icse/Bertolino07} (for a good survey of where software testing research is and wants to go, and some useful basic questions).
\end{itemize}

\section{Could Haves}

\begin{itemize}
\item ``Can a Machine Design?''\cite{doi:10.1162/07479360152681083} contains a really nice account of different modes of human/computer codesign,
and how having the computer make things for a human to correct is much less stressful than having the computer correct the human.
\item ``When and what to automate in software testing? {A} multi-vocal literature review''\cite{DBLP:journals/infsof/GarousiM16}---this
seems a natural fit.
\end{itemize}

\bibliography{references}{}
\bibliographystyle{acm}


\end{document}
