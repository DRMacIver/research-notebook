\documentclass[a4paper,oneside]{book}

%% Language and font encodings
\usepackage[english]{babel}
\usepackage[utf8x]{inputenc}
\usepackage[T1]{fontenc}

%% Sets page size and margins
\usepackage[a4paper,top=3cm,bottom=2cm,left=3cm,right=3cm,marginparwidth=1.75cm]{geometry}

%% Useful packages
\usepackage{amsmath}
\usepackage{amssymb}
\usepackage{amsthm}
\usepackage{graphicx}
\usepackage[colorinlistoftodos]{todonotes}
\usepackage[colorlinks=true, allcolors=blue]{hyperref}
\usepackage{listings}


\usepackage{mathtools}
\DeclarePairedDelimiter{\ceil}{\lceil}{\rceil}
\DeclarePairedDelimiter{\floor}{\lfloor}{\rfloor}

\newtheorem{definition}{Definition}[section]
\newtheorem{proposition}{Proposition}[section]
\newtheorem{lemma}{Lemma}[section]
\newtheorem{corollary}{Corollary}[section]
\newtheorem{theorem}{Theorem}[section]
\newtheorem{example}{Example}[section]
\newtheorem{claim}{Claim}[section]


% lstlisting macros
\lstset{
  language=Python, % choose the language of the code
  showspaces=false, % show spaces adding particular underscores
  showstringspaces=false, % underline spaces within strings
  showtabs=false, % show tabs within strings adding particular underscores
  tabsize=8, % sets default tabsize to 2 spaces
  captionpos=b, % sets the caption-position to bottom
  mathescape=true, % activates special behaviour of the dollar sign
  basicstyle=\footnotesize\tt,
  columns=fullfelxible,
  xleftmargin=2em,
  %commentstyle=\rmfamily\itshape,
  morekeywords={},
  escapeinside={(*@}{@*)},
  numbers=left
}
